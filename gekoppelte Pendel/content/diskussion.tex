\section{Diskussion}
Für die erste Pendellänge folgen die Messwerte:
\begin{align*}
T_{+} = 1.806\pm 0.005\si{\s}\\
T_{-} = 1.669\pm 0.005\si{\s}\\
T_{S} = 21.099 \pm 0.124\si{\s}\\
T_\text{S, theoretisch}= 22.141 \pm 1.444\si{\s}\\
\omega_{+} = 3.478\pm 0.006\si{\Hz}\\
\omega_{-} = 3.763\pm 0.007\si{\Hz}\\
\omega_{S}=-0.284\pm 0.009\si{\Hz}\\
\omega_\text{+, theoretisch} = 3.449\pm 0.016 \si{\Hz}\\
\omega_\text{-, theoretisch} = 3.476\pm 0.056 \si{\Hz}\\
\omega_\text{S, theoretisch} = -0.027 \pm 0.061\si{\Hz}\\
\kappa = 0.077 \pm 0.005
\end{align*}
Und für die zweite Pendellänge:
\begin{align*}
T_{+} = 1.419\pm 0.005\si{\s}\\
T_{-} = 1.262\pm 0.008\si{\s}\\
T_{S} = 10.513\pm 0.161\si{\s}\\
T_\text{S, theoretisch} = 11.416\pm 0.754\si{\s}\\
\omega_{+} = 4.428\pm 0.011\si{\Hz} \\
\omega_{-} = 4.98\pm 0.014\si{\Hz}\\
\omega_{S} = -0.54 \pm 0.018\si{\Hz}\\
\omega_\text{+, theoretisch} = 4.563 \pm 0.023\si{\Hz}\\
\omega_\text{-, theoretisch} = 4.617 \pm 0.025\si{\Hz}\\
\omega_\text{S, theoretisch} = -0.054 \pm 0.034\si{\Hz}\\
\kappa = 0.117 \pm 0.007
\end{align*}
Generell lässt sich sagen, dass wie erwartet die Schwingungsdauern der gegenphasigen Schwingung kürzer waren. Allerdings haben war auch zu erwarten, hohe 
Messunsicherheiten festzustellen und somit Diskrepanzen zu den theoretisch ermittelten Werten. Dies liegt insbesondere am menschlichen Fehler, da es nicht
leicht ist, die maximale Auslenkung genau zu treffen; hinzu kommt, dass bei der Anregung in gleich- oder gegenphasige Mode kaum derselbe Winkel getroffen werden
konnte und somit die Mode nicht perfekt angeregt werden konnte. Außerdem ist unausweichlich auch immer eine geringe Bewegung zur Wand hin und von der Wand weg hinzu 
gekommen, wodurch ein Teil der Energie der Pendelbewegung verloren gegangen ist und kein perfekter harmonischer Oszillator vorlag, so wie in der Theorie
angenommen. Reibung ist natürlich in der Praxis nicht zu vernachlässigen, die in die theoretische Rechnung auch nicht eingeflossen ist. So kämen noch Einflüsse
von Luftreibung hinzu, sowie die Reibung der Drehachse.\\
Auch schon die Kleinwinkelnäherung $sin(x) \approx x$ für kleine $x$ ist ungenau, da das Pendel nicht immer innerhalb der Toleranz ausgelenkt wurde und sich somit
Unterschiede zur Theorie ergeben.

\subsection{Schwebungsdauer}
Beim Vergleich fällt ein verhältnismässig großer Unterschied von auf, der nicht innerhalb der Fehler liegt. 
Dafür gibt es verschiedene Gründe. Zum einen ist die Messung der Schwebungsdauer nicht komplett genau möglich, da es kaum möglich ist, die Zeit genau zu stoppen,
und da 10 Messungen diese statistische Messunsicherheit nicht gänzlich aufheben können, ist so der Fehler für den gemessenen Wert recht hoch.
Zum anderen ist auch die Bestimmung über die theoretische Formel nicht fehlerfrei, da diese die Messwerte für $T_{+}$ und $T_{-}$ verwendet und auch diese mit einer
Messunsicherheit behaftet sind. Es ist besonders schwer, bei der Auslenkung in gleich- oder gegenphasiger Mode denselben Winkel zu treffen. Dadurch ist diese Mode
nicht einwandfrei angeregt und die Messung daher ungenau.\\
Hinzu kommt, dass die Pendel nicht genau auf einer Länge eingestellt sind, nur gemäß der Aufgabenstellung so, dass die Unterschiede in den Schwingungsdauern geringer sind
als die berechnete Standardabweichung. Durch die Bestimmung der Länge mit einem einfachen Maßband liegt auch eine gewisse systematische Messunsicherheit vor.
Gemeinsam ergeben diese Unsicherheiten dann den Unterschied in der Schwebungsdauer.\\
Bei der zweiten Pendellänge ergeben sich:
Da beim Messwert im Falle eines maximalen Fehlers $T_{S}  10.674\si{\s}$ gilt und beim Theoriewert für maximalen Fehler in die andere Richtung 
$T_\text{S, theoretisch} = 10.662\si{\s}$, überschneiden sich die Fehlerbereiche der Messwerte geringfügig. Es gelten dennoch dieselben Schwierigkeiten wie oben 
diskutiert.

\subsection{Frequenzen}
Auffällig ist die große Diskrepanz zwischend den Werten der Schwebungsfrequenz in Theorie und Messung bei beiden Pendellängen. Die anderen Werte liegen nah beieinaner, wenn auch nicht innerhalb
der Unsicherheit. 
Die oben beschriebenen Fehler und Unsicherheiten fliessen auch in die Frequenzen ein, sowohl bei den theoretischen Werten durch den Kopplungsgrad als auch bei den 
Messwerten durch die Unsicherheiten in der Schwingungsdauer. Wäre die Kopplungskonstante der Feder bekannt, könnte der Theoriewert genauer bestimmt werden. 


\subsection{Kopplungsgrade}
Für beide Kopplungsgrade ist die Messunsicherheit gering. Dabei ist die Kopplung bei der zweiten Pendellänge größer, was bei kurzem Nachdenken allerdings auch plausibel erscheint.
Die Massen befinden sich weiter oben, die Feder allerdings auf derselben Höhe; wenn die Pendel ausgelenkt sind, dehnt sich die Feder also früher aus, was zu einer höheren
Kopplung führt. Der Unterschied ist allerdings nicht groß genug, um einen wirklichen Unterschied beim Verhältnis von $T_{+}$ und $T_{-}$ sichtbar zu machen.