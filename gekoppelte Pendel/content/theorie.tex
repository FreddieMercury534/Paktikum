\section{Theorie}
\label{sec:Theorie}
\subsection{Das Fadenpendel}
    Ein Fadenpendel besteht aus einem Pendelkörper mit Masse m, der idealerweise 
    an einem masselosen Faden; in unserem Fall an einer Metallstange befestigt ist.
    Wird der Pendelkörper aus der Ruhelage ausgelenkt, so wirkt die 
    Gravitation $\vec{F}=m\cdot \vec{g} $ als rücktreibende Kraft . Dadurch entsteht  
    ein Drehmoment $M= D_p \cdot \sin{\theta}$ mit $D_p= m\cdot g\cdot l$. Für kleine Auslenkungen 
    kann die Kleinwinkelnäherung $\sin{\theta}=\theta$ verwendet werden. %\nocite{wiki}
    Unter Berücksichtigung des Trägheitsmomentes $J=m\cdot l²$ ergibt sich folgende Bewegungsgleichung 
    für den Pendelkörper
    \begin{equation}
        D_p \cdot \theta+J\cdot \ddot \theta=0
    \end{equation}
    Für kleine Winkel kann das Fadenpendel also als harmonischer Oszillator angenähert werden.
    Die Lösung dieser Gleichung lautet
    \begin{equation}
        \theta=a\cdot \sin{\omega t} + b\cdot \cos{\omega t}
    \end{equation}
    mit Schwingungsfrequenz
    \begin{equation}
        \omega= \sqrt{\dfrac{D_p}{J}}=\sqrt{\dfrac{g}{l}}
    \end{equation}
    Bei kleinen Auslenkungen haben daher weder die Masse des Pendelkörpers noch der
    Auslenkwinkel Einfluss auf die Periodendauer $T=\dfrac{2\pi}{\omega}$.

\subsection{Gekoppelte Schwingungen}
    Im Folgenden wird immer von zwei identischen Pendeln ausgegangen. Durch die 
    Kopplung von zwei Fadenpendeln durch eine Feder kann Energie des einen 
    Fadenpendels auf das andere übertragen werden. Deswegen wirken auf die Pendelkörper
    die zusätzlichen Drehmomente $M_1=D_F(\theta_2-\theta_1),\ M_2=D_F(\theta_1-\theta_2)$ 
    Die daraus resultierenden Differentialgleichungen lauten (mit Kleinwinkelnäherung):
    \begin{align}
         D_p \cdot \theta_1+J\cdot \ddot \theta_1=D_F(\theta_2-\theta_1)\\
         D_p \cdot \theta_2+J\cdot \ddot \theta_2=D_F(\theta_1-\theta_2)
    \end{align}
    In Abhängigkeit von den Anfangsbedingungen $\theta(0)$ und $\dot \theta(0)$ gibt es 
    verschiedene Schwingungsarten. Unterschieden wird vor allem zwischen den zwei Eigenmodi 
    (gleichsinnig und gegensinnig) und anderen Schwingungsarten. 

    \subsubsection{Gleichsinnige Schwingungen}
        Sind die Anfangsbedingungen (dh. Auslenkung und Geschwindigkeit) der gekoppelten
        Pendel exakt gleich, wird von gleichsinniger Schwingung gesprochen.
        Es findet keine Energieübertragung zwischen den beiden Fadenpendeln statt, da die 
        Feder im Idealfall zu keinem Zeitpunkt ge- oder entspannt wird. Die Schwingungsfrequenz
        stimmt mit den Eigenfrequenzen 
        \begin{equation}
            \omega_+= \sqrt{\dfrac{D_p}{J}}=\sqrt{\dfrac{g}{l}}
        \end{equation}
        der beiden Pendel überein, weshalb sich auch 
        die Periodendauer nicht durch die Kopplung ändert, weil dem System weder Energie 
        zugeführt noch entzogen wird. Sie beträgt daher ebenfalls 
        \begin{equation}
            T_+ = 2\pi \sqrt{\dfrac{l}{g}}
        \end{equation}
    \subsubsection{Gegensinnige Schwingungen}
        Schwingungen, bei denen die Auslenkungen zu Beginn $\theta_1=-\theta_2$ betragen, werden gegensinnige
        Schwingungen genannt. In dieser Eigenmode werden die rücktreibenden Kräfte der einzelnen Pendel 
        durch die rücktreibenden Kräfte der Feder um jeweils den gleichen Betrag verstärkt. Die 
        Bewegungsgleichung für beide $\theta$ lautet daher
        \begin{equation}
            D_p \cdot \theta +J\cdot \ddot \theta =D_F(2\theta)\Leftrightarrow 
            (2 D_p D_F) \theta +J\cdot \ddot \theta =0
        \end{equation}
        Die dadurch entstehende Schwingung ist symmetrisch und besitzt aufgrund der größeren 
        rücktreibenden Kräfte eine höhere Frequenz
        \begin{equation}
            \omega_-= \sqrt{\dfrac{D_p}{J}}=\sqrt{\dfrac{g}{l}+\dfrac{2 k}{l}}
        \end{equation}
        Dadurch verringert sich auch die Periodendauer
        \begin{equation}
            T_- = 2\pi \sqrt{\dfrac{l}{g+2k}}
        \end{equation}
        mit Federkonstante $k$.
    \subsubsection{Gekoppelte Schwingungen}
        Wird zu Beginn das eine Pendel ausgelenkt, das andere jedoch in der Ruheposition 
        belassen, so wird die Energie vollständig langsam hin und her übertragen, so dass 
        eines der Pendel still steht, sobald die Amplitude des anderen Pendels maximal wird.
        (Dies gilt zumindestens, wenn die Rückstellkraft der Feder gering ist im Vergleich 
        zur Rückstellkraft der Pendel). Die Schwingung ist darstellbar als Überlagerung von 
        zwei harmonischen Schwingungen $\theta_+$ und $\theta_-$. Die Schwebungsdauer und 
        die Schwebungsfrequenz berechnen sich dann als 
        \begin{equation}
            T_S = \dfrac{T_+\cdot T_-}{T_+ - T_-} \text{und} \omega_S=\omega_+ - \omega_-
        \end{equation}
        Um die Kopplung der beiden Fadenpendel zu beschreiben wird die Kopplungskonstente
        \begin{equation}
            K=\dfrac{\omega_-^2 - \omega_+^2}{\omega_-^2 + \omega_+^2}=
            \dfrac{T_+^2 - T_-^2}{T_+^2 + T_-^2}
        \end{equation}
        verwendet.