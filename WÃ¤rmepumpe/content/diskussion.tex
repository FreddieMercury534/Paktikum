\section{Diskussion}
\label{sec:Diskussion}

\subsection{Temperaturverläufe}
Vom den Temperaturverläufen würden wir erwarten, dass sie genau
gespiegelt voneinander verlaufen. Im Plot erkennt man allerdings,
dass die Kurve des Reservoir 2 flacher verläuft. Da die Wärmepumpe 
vorrangig eingesetzt wird, um Reservoir 1 kostengünstig zu erhitzen,
wäre es möglich, dass mehr Wert auf eine hinreichende Isolierung des 
zu erhitzenden Reservoirs gelegt wurde. Des weiteren
entspricht der Temperaturverlauf für das Reservoir 1 der Ausgleichskurve
nahezu einwandfrei, während der Temperaturverlauf des kälteren Reservoirs
geringe Abweichungen aufweist (siehe \ref{fig:temp}).
Anstelle von sprunghaften Schwankungen liegen allerdings kontinuierliche,
wellenförmige Schwankungen vor, was die Annahme nahe legt, dass 
es sich nicht um Messunsicherheiten des Thermometers handelt, sondern
dass die Temperatur tatsächlich etwas geschwankt hat, beispielsweise
durch inhomogenes Vermischen der Wasserschichten im Reservoir oder durch
nicht ausreichende Isolierung der Gefäße.\\
\subsection{g) Güteziffer}
Die reale Güteziffer weicht stark vom Idealwert ab. Da allerdings beim 
Idealwert von einem reversiblen Prozess ausgegangen wird, der in der 
Praxis und so auch der vorliegenden Wärmepumpe definitiv nicht umsetzbar
ist, ist das nicht verwunderlich. Das größte Problem stellt dabei die 
Isolierung dar, weil bei nicht ausreichender Isolierung Energie verloren 
geht, es sich also nicht um ein abgeschlossenes System handelt.
